\section*{Exercise 1.14}

Let's analyze the number of steps. For the case when we count change for one kind of coins, it can easily be noticed that we need to perform $\Theta(n)$ steps, where $n$ is the amount of coins. Now let's consider the case where the number of the kinds of coins equals 2. The following diagram represents the pattern of computation:

\[
	\begin{diagram}
		\node[2]{cc\;n\; 2} \arrow{sw}
		\arrow{se}
		\\
		\node{cc\;n\;1} \node[2]{cc\;(n-const)\;2}
	\end{diagram}
\]

We know that the left side has order of growth equal to $n$. And we can see that the right side will produce $n$ approximately $n/const$ times. Thus, the number of steps used by this process approximates to $n+n+n\dots$ $n/const$ times, which is $\Theta(n^2)$.

We can see some pattern here, but to be sure what it is, let's move on to the next case, where the number of kinds of coins equals $3$. The diagram for this case looks like this:

\[
	\begin{diagram}
		\node[2]{cc\;n\; 3} \arrow{sw}
		\arrow{se}
		\\
		\node{cc\;n\;2} \node[2]{cc\;(n-const)\;3}
	\end{diagram}
\]

The same reasoning takes place here. We evaluate the left side, which has order of growth equal to $\Theta(n^2)$ and we need to add this number of steps $n/const$ times, therefore, the overall order of growth is $\Theta(n^3)$. Thus, we can make a conclusion that the amount of steps required is $n^k$ where n is the amount of money and $k$ is the number of kinds of coins.
