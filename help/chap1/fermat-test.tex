\section*{The Fermat test}

After watching some videos on youtube, we come to the conclusion that the difference $a^p - a$ should be divisible by $p$ if $p$ is prime. It easily transforms into the following equation:

\[
a^p - a = xp
\]

where $x$ is an integer. This equation can be rewritten in the following way:

\[
a^p = xp + a
\]

where $a$ is the remainder of division of $a$ by $p$. We can rewrite it so:

\[
\text{rem}\left(\frac{a^p}{p} \right) = a
\]

Now we need to consider 2 different cases: the first one when the exponent $p$ (let's denote it $2k$) is even:

\[
\text{rem} \left(\frac{a^{2k}}{n}\right) = \text{rem}\left(\left(\text{rem}\left(\frac{a^k}{n}\right)\right)^2 /n \right)
\]

And when the exponent $p$ (let's denote it $k$) is odd:

\[
\text{rem}\left(\frac{a^k}{n}\right) = \text{rem}\left(\left(\text{rem}\left(\frac{a^{k-1}}{n}\right) \text{rem}\left(\frac{a}{n}\right)\right) / n \right)
\]

We chose $a$ to be less than $n$. This means that the remainder of division of $a$ by $n$ is always $a$. Therefore, the equation above can be rewritten in the following way:

\[
\text{rem}\left(\frac{a^k}{n}\right) = \text{rem}\left(\left(\text{rem}\left(\frac{a^{k-1}}{n}\right) a \right) / n \right)
\]

Now, let's consider the final condition in the process. When we reach 1 at the end of the computation, it means that we get 1, as the remainder of 1 divided by any number will always be 1 (and we surely don't divide by 1 through the whole process).

All the reasonings above give us the way to construct the procedure for implementing the Fermat test.