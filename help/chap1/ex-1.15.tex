\section*{Exercise 1.15}

Let's prove this formula first:

\begin{equation}
	\sin x = \sin \frac {r} {3} - 4 \sin^3 \frac {r} {3}
\end{equation}

We'll use this formulae

\begin{empheq}[left=\empheqlbrace]{align}
	\sin (a+b) = \sin a \cos b + \cos a \sin b\\
 	\cos (a+b) = \ cos a \cos b - \sin a \sin b
\end{empheq}

Now, let's expand the following equation:

\begin{gather}
	\sin(2a+a) = \sin 2a \cos a + \cos 2a \sin a = \\
	\sin 2a \cos a + \left(1 - 2 \sin^2 a\right) \sin a = \\
	\sin 2a \cos a + \sin a - 2 \sin^3 a = 
	2 \sin a \cos^2 a + \sin a - 2 \sin^3 a = \\
	2 \sin a \left(1 - sin^2 a\right) + \sin a - 2 \sin^3 a = 
	3 \sin a - 4 \sin^3 a
\end{gather}

From this equation it's easy to see that

\begin{equation}
	\sin x = 3 \sin \frac {x} {3} - 4 \sin^3 \frac {x} {3}
\end{equation}

Let's try to count the amount of times the procedure p applied. At each step the angle divides by 3. So we have the following process:

\[
\begin{aligned}
	12.15 / 3 = 4.05 \\
	4.05 / 3 = 1.35 \\
	1.35 / 3 = 0.45 \\
	0.45 / 3 = 0.15 \\
	0.15 / 3 = 0.5
\end{aligned}
\]

$0.5$ is less than $0.1$ so our condition has been reached. From here we see that the procedure is applied 5 times.

It's time to estimate the order of growth in space and number of steps used by the process. To do this let's take a note that each step $a$ divides by $3$ n times until it reaches $0.1$.

\begin{equation}
	\underbrace{a / 3 / 3 / 3 \dots}_{n\ times}
\end{equation}

Oh, boy, we can clearly see from this formula that the process grows logarithmically.

\[
	\begin{aligned}
		\frac {a} {3^n} &< 0.1 \\
		a &< 0.1 3^n \\
		10 a &< 3^n \\
		\log _3 10 a &< n \\
		n &> \log _3 10 a \\
	\end{aligned}
\]

We see that logarithm grows logarithmically. The space is proportional to the amount of steps required and therefore also grows as $\log _3 10 a$, i.e. has the order of growth $\Theta(\log a)$.
