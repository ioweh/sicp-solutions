\section*{Exercise 1.18}

We better use the invariant quantity for designing such a procedure. It can be described by the following expression:

\[
ab + k
\]

Suppose that at the first step $b$ is even and can be divided by 2 without a remainder.

\[
(2a_1)(b_1 / 2) + k_1
\]

Note that $k_1$ is still 0 at this step. We can rewrite the expression in the following way:

\[
a_2 b_2 + k_2
\]

Now let's assume that $b_2$ is odd and use the following transformation:

\[
a_2 + a_2 (b_2 - 1) + k_2
\]

Or, rewriting it:

\[
a_2 (b_2 - 1) + (a_2 + k_2) 
\]

Let's rewrite this once again:

\[
a_3 b_3 + k_3
\]

And so on... At the step before the last step we have:

\[
a_{n-1} b_{n-1} + k_{n - 1}
\]

Note that $b_{n - 1}$ at this step equals 1, which means that it's odd. Therefore, we apply the following transformation:

\[
a_{n - 1}(b_{n - 1} - 1) + (a_{n-1} + k_{n - 1})
\]

From here we get:

\[
k_n = a_{n - 1} + k_{n - 1}
\]

Note that the result of the expression doesn't change from step to step. So the answer is given by $k_n$ received in the last step of the process. All of the notions easily transform to the following code:

\begin{verbatim}
(define (fast-mul-iter-step a b k)
  (cond ((= b 0) k)
	((odd? b) (fast-mul-iter-step a (- b 1) (+ a k)))
	(else (fast-mul-iter-step (* 2 a) (/ b 2) k))))
\end{verbatim}

This is an iterative process that uses a logarithmic number of steps and has $\Theta(1)$ space complexity.

