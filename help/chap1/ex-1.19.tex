\section*{Exercise 1.19}

Let's actually apply the transformation $T_{pq}$ twice. We start with the pair $a_0 b_0$. The first step is:

\begin{align*}
a_1 &= bq + aq + ap \\
b_1 &= bp + aq
\end{align*}

In the second step, we receive the following:

\begin{align*}
\mathllap{a_2} &= b_1 q + a_1 q + a_1 p = (bp + aq) q + (bq + aq + ap) q + (bq + aq + ap) p\\
&= bpq + aq^2 + bq^2 + aq^2 + apq + bpq + apq + ap^2\\
&= b \underbrace{(2pq + q^2)}_{q^\prime} + a \underbrace{(2pq + q^2)}_{q^\prime} + a \underbrace{(p^2 + q^2)}_{p^\prime}\\
b_2 &= b_1 p + a_1 q = (bp + aq)p + (bq + aq + ap)q\\
&= bp^2 + apq + bq^2 + aq^2 + apq = b\underbrace{(p^2 + q^2)}_{p^\prime} + a\underbrace{(2pq+q^2)}_{q^\prime}
\end{align*}

So, we see from this transformation that the coefficients are:

\begin{align*}
p^\prime &= p^2 + q^2\\
q^\prime &= 2pq + q^2
\end{align*}

By applying this transformation, we reduce the amount of the numbers we need to count twice. Because we can count $a_4, b_4$ on the second step, once we applied this transformation, $a_6, b_6$ on the third and so on. Now, if we apply such a transformation once again, we again will reduce the amount of the numbers we need to count twice. So, we could count $a_4, b_4$ right after the second application of the transformation, starting with $a_0, b_0$. And after that we could count the pair $a_8, b_8$ and so on. The procedure does actually run in a logarithmic number of steps.

